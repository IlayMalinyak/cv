\documentclass[11pt]{article}
\usepackage[a4paper,margin=1in]{geometry}
\usepackage{fontawesome5}
\usepackage{xcolor}
\usepackage{hyperref}
\usepackage{fontawesome5}   % For icons
\usepackage{xcolor}         % For colors
\usepackage{hyperref}       % For links

\hypersetup{
    colorlinks=true,
    urlcolor=black,
    linkcolor=black,
}

\usepackage{parskip}
\usepackage{enumitem}

\begin{document}

\begin{center}
    {\LARGE \textbf{Ilay Kamai}}\\
    \textit{Machine Learning \& Astrophysics} \\
    \vspace{0.5em}
    \href{mailto:ilay.kamai@campus.technion.ac.il}{\textcolor{black}
    {\faEnvelope}\ \textcolor{black}{ilay.kamai@campus.technion.ac.il}} \quad
    \href{https://github.com/IlayMalinyak}{\textcolor{black}
    {\faGithub}\ \textcolor{black}{IlayKamai}} \quad
    \href{https://orcid.org/my-orcid?orcid=0009-0008-5080-496X}{\textcolor{green!50!black}{\faOrcid}\ 0009-0008-5080-496X}
\end{center}

\vspace{1em}


\section*{Profile}
Ph.D. student in the Faculty of Physics at the Technion. \\ Interested in the intersection between data science and astrophysics and specifically machine learning approaches for stellar astrophysics.

\section*{Education}
\begin{itemize}[leftmargin=*]
 \item \textbf{Ph.D. in Physics and CS}, Technion – Israel Institute of Technology \\
  December 2024 - \\
  Advisors: Professor Hagai B. Perets, Professor Alex Bronstein
  \item \textbf{M.Sc. in Physics}, Technion – Israel Institute of Technology \\
  October 2022 -- November 2024 (Cum Laude). \\ Advisor: Professor Hagai B. Perets

  \item \textbf{B.Sc. in Physics}, Minor in Computer Science, Hebrew University of Jerusalem \\
  October 2016 -- July 2020

\end{itemize}

\section*{Publications}
\begin{itemize}[leftmargin=*]
  \item Kamai, I., Bronstein, A. M.,Perets, H. B. (2025). \textit{Machine-learning inference of stellar properties using integrated photometric and spectroscopic data}. arXiv:2507.10666. The Astrophysical Journal.
  \item Kamai, I., Perets, H. B. (2025). \textit{Too Fast to Be Single: Tidal Evolution and Photometric Identification of Stellar and Planetary Companions}.
  arXiv:2503.03839. Open Journal of Astrophysics.
  \item Kamai, I., Perets, H. B. (2024). \textit{Accurate and Robust Stellar Rotation Periods Catalog for 82,771 Kepler Stars Using Deep Learning}. arXiv:2407.06858. The Astronomical Journal.
\end{itemize}

\section*{Conference and invited talks}
\begin{itemize}[leftmargin=*]
\item \textbf{LightPred- a deep learning model for stellar properties predictions}, ML4Astro, Catania, Jul 2024.
\item \textbf{Stellar properties prediction using deep learning}, AsCos IV, Weitzman institute, Aug 2024.
\item \textbf{Machine learning approach for stellar light curve analysis}, BIU, Jan 2025.
\item \textbf{Unfolding Stellar Properties with Deep Learning -
 Multi-Modal approach for Spectroscopy and Photometry
} (poster), UniversAI, Athens, Jun 2025.
\item \textbf{Key Stellar Parameter Predictions with Multi-Modal Neural Networks: Extending Deep Learning to Spectroscopy and Photometry}, AstroAI, Center for Astrophysics (CfA), Harvard, Jul 2025. See the \href{https://www.youtube.com/watch?v=Xz4AwGFamLY&t=461s}{\textcolor{blue}{recording}}.

\item \textbf{DESA - a Multi-Modal approach for Stellar Astrophysics} (poster), Workshop on Machine Learning for Astrophysics (ICML 2025), Vancouver, Jul 2025.
\item \textbf{Multi-modal approach for stellar astronomy}, Princeton, Jul 2025.
\item \textbf{Multi-modal approach for stellar astronomy}, Center of Computational Astrophysics (CCA), Flatiron Institute, NYC, Jul 2025.
\end{itemize}

\section*{Employment History}
\textbf{R\&D Physicist and Algorithms Developer}, Alpha Tau Medical \\
November 2020 -- October 2022

\end{document}

